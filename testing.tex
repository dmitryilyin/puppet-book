\chapter{Тестирование модулей и манифестов}

Ошибки в манифестах Puppet часто могут привести как к их неработоспособности, так и к нарушению работы управляемых систем и потере данных. Поэтому перед запуском новых версий их необходимо сначала проверять. Рассмотрим как можно организовать тестирование.

\section{Методы тестирования}

Методика тестирования может сильно отличаться в зависимости от того, в каких условиях и с какими целями проводятся тесты. Рассмотрим, какими средствами можно выполнять проверку модулей и манфестов.

\subsection{Полный функциональный тест}

Хотя проведение тестов на работающей системе обычно может привести к нарушению её работы и не рекомендуется, иногда может не быть возможности использовать другие методы. В этом случаем можно проводить тестирование на одном из рабочих серверов, или, лучше, на специальной тестовой системе.

Для того, чтобы разработка и отладка новых манифестов не влияла на работу остальных серверов, можно использовать механизм окружений на управляющей системе. Кроме основного набора модулей и манифестов, которые используются всеми серверами, создается еще один дополнительный тестовый набор, который может использоваться только тестовой системой. Изменения в тестовом наборе никак не влияют на работу остальных серверов, но позволяют легко проводить отладку, использую для этого единую инфраструктуру.

Чтобы создать тестовое окружение добавляем в конфигурационный файл \textbf{puppet.conf} на управляющем сервере следующие строчки:

\begin{verbatim}
[testing]
  modulepath = $confdir/environments/testing/modules
  manifest = $confdir/environments/testing/manifests/site.pp
\end{verbatim}

Мы указали, что для окружения \texttt{testing} модули и манфесты будут находиться в отдельном каталоге \texttt{environments/testing}, котоый находтся рядом с основным набором модулей.

На тестовой системе теперь можно запустить агент с дополнительным параметром \texttt{--environment testing}, чтобы он получил каталог, собранный из тестовой ветки.

Кроме \texttt{testing} моджно создать много дополнительных окружений, например разделив резработку и тестирование, или эту же технологию можно использовать для разделения между серверами разных подразделений компании. Окружение по умолчанию называется \texttt{production}, и оно используется, если не создано никаких дополнительных окружений.

В процессе разработки изменения вносятся в тестовую ветку и на одной из систем, которая используется как тестовая, можно проверить их результат. Самый простой способ --- сначала запустить Puppet в режиме \textbf{noop}, с котором никакие изменения не будут на самом деле произведены, а только будет показано, что агент бы сделал. Запустить агент в таком режиме можно так:

\begin{verbatim}
puppet agent --environment=testing --test --noop
\end{verbatim}

Будет выведен список всех изменений, которые были бы сделаны, если бы агент был запущен без ключа \texttt{-{}-noop}. На этом этапе уже можно заметить ошибку, если в списке есть не те изменения, которые хотел выполнить разработчик. Этот метод обычно позволяет обнаружить только логические и смысловые ошибки. Пока нельзя сказать будут ли правильно выполнены все эти изменения и в правильном ли порядке.

Если список изменений выглядит правильным, можно убрать параметр \texttt{-{}-noop} и применить все изменения. При запуске только с параметром \texttt{-{}-test} будут выводиться все выполняемые изменения и ошибки, если они происходят. Такая проверка иногда называется \textit{Smoke Test}, потому что она позволяет быстро проверить не сломали ли систему последние изменения в модуле или манифесте.

\begin{verbatim}
puppet agent --environment=testing --test
\end{verbatim}

Если тест новой версии прошёл успешно на всех тестовых системах, или после исправления всех ошибок и проблем, то можно выложить новую версию из тестового окружения в основное, чтобы оно было применено на всех серверах. Если изменения всё же потенциально могут привести к проблемам, то часто используют поэтапный переход, когда сначала изменения применяются на одной группе серверов, потом, если не возникло проблем, на следующей, и так пока все сервера не будут обновлены.

Использование системы контроля версий позволяет легко вносить изменения, например, сначала в ветку для разработчиков, потом передавать их на тестирование в тестовую ветку, затем, если тесты прошли успешно, передать изменения в основную ветку. Кроме этого можно легко отменить последние изменения и вернуться к более ранней версии, которая работает правильно, если появились непредвиденные проблемы.

Такая методика тестирования уместна, когда Puppet используется для поддержки большой инфраструктуры реальных серверов, которая слишком сложна, чтобы можно было сделать небольшую тестовую модель, которая вела бы себя достаточно похоже на реальную систему.

Тестирование на реальной системе позволяет определить работоспособность манифестов именно в тех условиях, в которых они будут использоваться, или для нескольких ситуаций, которые присутствуют в поддерживаемой системе. Если модуль более универсальный, то можно легко не обнаружить ошибки, которые проявляются в других ситуациях или при редком сочетании параметров.

\subsection{Модульные тесты}

При разработке отдельного модуля часта приходится проверять его работы в разных окружениях с разными входными данными. Использование реальной инфраструктуры обычно позволяет проверить работу только в некоторых ситуациях.

Если не требуется проверка всего набора модулей и манифестов сразу, то можно использовать модульные тесты, которые позволяют проверить работоспособность каждого модуля отдельно, используя при этом специальные тестовые окружения, позволяющие быстро смоделировать разные состояния системы.



\subsection{Unit тесты}

\subsection{Этапы тестирования}