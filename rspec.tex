\chapter{Unit тесты Rspec}

\section{Разработка через тестирование}

Rspec --- фреймворк для тестирования поведения классов Ruby. Тестирование поведения это практика программирования, при которой перед написанием самой программы создаётся набор тестов, описывающих её поведение в разных режимах и примерах использования. Эти тесты получаются болше похожими на спецификацию или даже документацию, описывающую работу программы.

После написания тестовс они, естественно, не проходят, потому что программы еще не существует и можно переходить к этапу написания самой программы так, чтобы все тесты слали успешными. Обычно процесс разработки построен итеративно. Сначала реализуются тесты для самых базовых возможностей программы, реализуются эти возможности, потом пишутся тесты для дополнительных более продвинутых возможностей и приступают к их реализации. Таким образом на каждой итерации получается работающий продукт, проходящий все тесты и готовый к использованию, но, возможно, имеющий еще не все возможности, которые планировалось. При каждой последующие итерации всегда выполняются все тесты и происходит обнаружение всех возможных регрессий.

Такой процесс разработки лежит в основе методики Agile и сильно отличается от классической модели, при которой сначала планируются все возможности, разрабатывается архитектура, реализуются все компоненты и потом тестируются, что делает довольно сложным внесение изменений в структуру проекта, и работающая версия получается только на последних этапах работы.

Тестирование поведения --- расширение концепции разработки через тестирование, при котором делается фокус не не структуру программы и соответствие работы каждого её компонента спецификации, а на поведение всей программы целиком и ее реакции на различные ситуации и исходные данные.

\section{Начало работы с Rspec}

Для начала установим Rspec. Это можно сделать при помощи gem.

\begin{verbatim}
gem install rspec
\end{verbatim}

Или при помощи пакетного менеджера вашего дистрибутива.

\begin{verbatim}
apt-get install ruby-rspec
yum install rubygem-rspec
\end{verbatim}

Для примера разработаем простую программу для сложения чисел. Программа должна складывать все переданные числа и выводить результат.

Описание поведения пишется при помощи специального предметно-ориентированного языка, реализованного на Ruby. При описании используется три ключевых слова: Дано, Если, Тогда. Первое слово --- задание условий, при которых производится тестирование, второе --- некоторое событие или входные данные программы, а третье --- результат, который должна вернуть программа при правильной работе.

Сначала опишем группу тестов при помощи слова \textbf{describe}. Весь связанный с ней блок будет входить в эту группу. Группы могут быть вложенными друг в друга и вместо слова \textbf{describe} может быть использовано слово \textbf{context}, чтобы подчеркнуть разделение тестов на относящиеся к разным условиям. Тесты описываются при помощи ключевого слова \textbf{it}. Описание тестов и контекстов обычно пишут простыми словами, которые были бы понятны не только программистам, но и людям не технических специальностей, но знакомыми с бизнес процессами, делая возможным их участие в составлении описаний программ.

\lstinputlisting[title={Spec для SumNumbers}]{code/rspec/SumNumbers/correct/SumNumbers_spec.rb}

Теперь напишем саму программу так, чтобы все тесты могли завершиться успешно. Создадимк Ruby класс с тремя методами: конструктором, устанавливающим начальное значение, методом, прибавляющем к этому значению число, и методом, возвращающим хранимое значение.

\lstinputlisting[title={Программа SumNumbers}]{code/rspec/SumNumbers/correct/SumNumbers.rb}

Теперь запустим тесты и посмотрим на результат. Тест можно запустить при помощи команды \textbf{rspec}. Укажим формат documentation, чтобы включить вывод всех выполненных тестов.

\begin{verbatim}
rspec -f doc SumNumbers_spec.rb
\end{verbatim}

\verbatiminput{code/rspec/SumNumbers/correct/result.log}

Теперь рассмотрим ситуацию, когда в программе есть ошибка, и тесты завершаются неудачно. Предположим, что был неправильно реализован метод-конструктор и он всегда устанавливает нулевое начальное значение суммы.

\lstinputlisting[title={Программа SumNumbers с ошибкой}]{code/rspec/SumNumbers/error/SumNumbers.rb}

Теперь снова запустим тесты и убедимся в том, что они не проходят.

\verbatiminput{code/rspec/SumNumbers/error/result.log}


\section{Совпадения и ожидания}

При проверке соответствия результатов выполнения программы правильному значению используются такие понятия как Expectations (Ожидания) и Matchers (Совпадения). Expectations устанавливаются автоматически при установке rspec и добавляют два метода каждому Ruby объекту, которые используются для сравнения полученного значения с тем, которое должно быть, или не должно быть.

\begin{verbatim}
foo.should eq(bar)
foo.should_not eq(bar)
\end{verbatim}

Можно использовать альтернативный синтаксис:

\begin{verbatim}
expect(foo).to eq(bar)
expect(foo).not_to eq(bar) 
\end{verbatim}

Аргументом для этих объектов должно быть Совпадение. Эти объекты определяют разные условия, которые могут ожидаться или не ожидаться при проверке. Среди встроенных совпадений можно найти равенство, неравенство, нахождение в массиве, принадлежность классу и многие другие.

\begin{description}
\item[Сайт проекта Rspec] \url{http://rspec.info}
\item[Документация] \url{https://www.relishapp.com/rspec}
\item[Проест на GitHub] \url{https://github.com/rspec/rspec}
\end{description}
