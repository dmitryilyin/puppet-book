\chapter{Unit тесты Rspec}

\section{Основы работы с Rspec}

\subsection{Технологии тестирование кода}

Rspec --- фреймворк \textit{Unit тестов} для программ, написанных на Ruby. Unit тесты позволяют программистам разделить программу на составные части и автоматически проверить корректность каждой из них. Для такой проверки требуется написание специальных тестов для каждой достаточно нетривиальной части программы. Тестирование позволяет легко находить ошибки и регрессии, которые могут возникнуть в процессе разработки.

Система Unit тестов уменьшает сложность поддержки программного кода и упрощает внесение изменений, помогая программистам обнаружить возможные побочные эффекты каждого изменения. Обычно Unit тесты создаются в процессе работы над программой, или даже после её написания. Чем большее число объектов и методов программы тестируется с разными возможными вариантами входных данных, тем выше \textit{покрытие тестами} кода.

Разработка через тестирование --- такая практика программирования, при которой перед написанием самой программы создаётся набор тестов, описывающих её работу, а разработка самой программы начинается уже после завершения работы над тестами. Это позволяет использовать тесты не только для автоматического поиска ошибок, но и как \textit{спецификацию} программы, или даже \textit{исполняемую документацию}. При этом можно говорить скорее о покрытии спецификации кодом, чем о покрытии кода тестами.

Такой процесс разработки лежит в основе методики \textbf{Agile} и сильно отличается от классической модели, при которой сначала планируются все возможности, разрабатывается архитектура, реализуются все компоненты и тесты для них, что делает довольно сложным внесение изменений в структуру проекта, и работающая версия получается только на последних этапах работы.

Процесс разработки по этой методике построен из большого количества коротких итераций. Сначала описываются самые базовые возможности программы, и пишется код, реализующий их, запускается тестирование и проверяется корректность этого кода. После этого спецификация расширяется все большим и большим числом функций, которые реализуются и снова запускается тестирование. Таким образом на каждой итерации получается работающий продукт, проходящий все тесты и готовый к использованию, но имеющий еще не все возможности, которые планировалось. При каждой последующие итерации всегда выполняются все тесты и происходит обнаружение возможных регрессий.

Тестирование поведения --- расширение концепции разработки через тестирование, при котором делается фокус не на описание структуры программы и соответствие работы каждого её компонента спецификации, а на поведение всей программы целиком и ее реакции на различные ситуации и исходные данные. Это позволяет говорить о всей программе как о некотором чёрном ящике, который ведёт себя определённым образом при передаче ему входных данных. Такой подход даёт возможность участвовать в процессе разработки спецификации людям, которые далеки от программирования, но хорошо знакомы с бизнес-процессами, которые должна реализовывать разрабатываемая программа.

Unit тесты Rspec могут использоваться и в классической методике разработки, и в разработке через тестирование. Язык, используемый при написании тестов, слишком близок к самому языку Ruby. Для описания возможностей программы на понятном не-программистам языке лучше использовать другой инструмент --- \textbf{Cucumber}.

\subsection{Создание спецификации для Ruby приложения}

Попробуем создать простое Ruby приложение по методике разработки через тестирование, написав сначала его спецификацию. Для начала установим сам Rspec. Это можно сделать при помощи gem.

\begin{verbatim}
gem install rspec
\end{verbatim}

Или при помощи пакетного менеджера вашего дистрибутива.

\begin{verbatim}
apt-get install ruby-rspec
yum install rubygem-rspec
\end{verbatim}

Разработаем простую программу для сложения чисел. Программа должна складывать все переданные ей числа и выводить результат.

Описание поведения пишется при помощи специального предметно-ориентированного языка. При описании используется три ключевых слова: \textbf{Дано}, \textbf{Если}, \textbf{Тогда}. Первое слово --- задаёт условия, при которых производится тестирование, например создаёт экземпляр класса. Второе --- некоторое событие или входные данные, которые передаются этой программе. А третье --- результат, который должна вернуть программа при правильной работе.

Сначала опишем группу тестов при помощи слова \textbf{describe}. Весь связанный с ней блок тестов будет входить в эту группу. Группы могут быть вложенными друг в друга и вместо слова \textbf{describe} может быть использовано слово \textbf{context}. Тесты описываются при помощи ключевого слова \textbf{it}.

\lstinputlisting[title={Spec для SumNumbers}]{code/rspec/SumNumbers/correct/SumNumbers_spec.rb}

Теперь напишем саму программу так, чтобы все тесты могли завершиться успешно. Создадим Ruby класс с тремя методами: конструктором, устанавливающим начальное значение, методом, прибавляющем к этому значению число, и методом, возвращающим хранимое значение.

\lstinputlisting[title={Программа SumNumbers}]{code/rspec/SumNumbers/correct/SumNumbers.rb}

Теперь запустим тесты и посмотрим на результат. Тест можно запустить при помощи команды \textbf{rspec}. Укажем формат documentation, чтобы включить вывод всех выполненных тестов.

\begin{verbatim}
rspec -f doc SumNumbers_spec.rb
\end{verbatim}

\VerbatimInput{code/rspec/SumNumbers/correct/result.log}

Теперь рассмотрим ситуацию, когда в программе есть ошибка, и тесты завершаются неудачно. Предположим, что был неправильно реализован метод-конструктор и он всегда устанавливает нулевое начальное значение суммы.

\lstinputlisting[title={Программа SumNumbers с ошибкой}]{code/rspec/SumNumbers/error/SumNumbers.rb}

Теперь снова запустим тесты и убедимся в том, что они не проходят.

\VerbatimInput{code/rspec/SumNumbers/error/result.log}

\subsection{Совпадения и ожидания}

При проверке соответствия результатов выполнения программы правильному значению используются такие понятия как Expectations (Ожидания) и Matchers (Совпадения). Expectations устанавливаются автоматически при установке Rspec и добавляют два метода каждому Ruby объекту, которые используются для сравнения полученного значения с тем, которое должно быть, или не должно быть.

\begin{verbatim}
foo.should eq(bar)
foo.should_not eq(bar)
\end{verbatim}

Можно использовать альтернативный синтаксис:

\begin{verbatim}
expect(foo).to eq(bar)
expect(foo).not_to eq(bar) 
\end{verbatim}

Аргументом для этих объектов должно быть Совпадение. Эти объекты определяют разные условия, которые могут ожидаться или не ожидаться при проверке. Среди встроенных совпадений можно найти равенство, неравенство, нахождение в массиве, принадлежность классу и многие другие.

\begin{description}
\item[Сайт Rspec] \url{http://rspec.info}
\item[Документация] \url{https://www.relishapp.com/rspec}
\item[GitHub] \url{https://github.com/rspec/rspec}
\end{description}

\section{Использование Rspec для проверки модулей Puppet}

\subsection{Подготовка к использованию тестов}

Тестирование поведения модулей и манифестов puppet не предусматривает их запуск на реальной или виртуальной системе, а описывает, какие ресурсы и с какими параметрами должны быть собраны в каталог, который затем был бы исполнен на управляемой системе. Таким образом проверяется логика, реализуемая модулем и манифестом посредством проверки соответствия результатов их работы эталонным результатам в указанных условиях с определёнными параметрами.

Например можно убедиться, что на Debian системе будет установлен пакет apache и создан файл \texttt{mysite.ru.conf} c описанием этого сайта в каталоге \texttt{sites-aviable}  и создана символическая ссылка на этот файл в \texttt{sites-enabled}. А при запуске на Red Hat системе вместо этого будет установлен пакет httpd и создан файл \texttt{mysite.ru.conf} в каталоге \texttt{conf.d}.

Такой тест совершенно не гарантирует, что все эти действия действительно будут выполнены правильно, он только проверяет правильность постановки задачи для исполнения на управляемой системе. Процесс может завершиться неудачно по множеству причин, которые не зависят от правильности написания этого манифеста.

Хотя использование методов разработки через тестирование может показаться бесполезным для не очень сложных модулей и манифестов, при увеличении количества управляемых ресурсов, появлении сложных условий и логики, поддержке многих конфигураций и версий операционных систем и, особенно, при дальнейшей доработке существующих манифестов отладка и тестирование станет очень сложным. Поэтому Unit тесты, хотя и не заменяют проверки на тестовых системах, позволят определить большинство логических ошибок, регрессий и несоответствий спецификации еще до начала тестирования. Кроме этого хорошо написанные тесты также выполняют функцию "исполняемой документации", описывая поведение каждого манифеста на языке, который легко понятен не только компьютеру, но и человеку.

Чтобы использовать Rspec для тестирования модулей Puppet нужно сначала установить пакет rspec-puppet, который можно найти здесь \url{http://rspec-puppet.com}. Кроме него потребуется установить сам puppet и дополнительный пакет puppetlabs\_spec\_helper.

\begin{verbatim}
gem install rspec-puppet
gem install puppetlabs_spec_helper
gem install puppet
\end{verbatim}

\subsection{Тестирование простого класса}

Теперь создадим простой модуль puppet и тесты для него. Предположим, что у нас есть модуль для управления файлом \textbf{/etc/issue}, состоящий из одного класса.

\begin{lstlisting}
class motd {
  file { '/etc/motd' :
    ensure  => present,
    owner   => 'root',
    group   => 'root',
    mode    => '0644',
    content => 'Hello!',
  }
}
\end{lstlisting}

Оформим этот класс как модуль, создав соответствующую структуру каталогов и положив этот класс в файл \textbf{init.pp} в каталоге manifests.

% \dirtree{%
% .1 modules.
% .2 motd.
% .3 manifests.
% .4 init.pp.
% .3 tests.
% .3 lib.
% .3 files.
% .3 templates.
% }

Теперь воспользуемся скриптом \textbf{rspec-puppet-init} перейдя в каталог этого модуля чтобы создать необходимую для работы Unit тестов структуру файлов и каталогов.

\begin{verbatim}
 + spec/
 + spec/classes/
 + spec/defines/
 + spec/functions/
 + spec/hosts/
 + spec/fixtures/
 + spec/fixtures/manifests/
 + spec/fixtures/modules/
 + spec/fixtures/modules/motd/
 + spec/fixtures/manifests/site.pp
 + spec/fixtures/modules/motd/manifests
 + spec/spec_helper.rb
 + Rakefile
\end{verbatim}

\begin{description}
\item[classes] В этом каталоге нужно размещать файлы, описывающие поведение классов и параметрических классов.
\item[defines] В этом каталоге нужно размещать описание поведения определений.
\item[hosts] Здесь размещаются описания работы модуля на разных системах.
\item[functions] Описания работы специальных функций, если модуль их содержит.
\item[fixtures] Здесь создаются символические ссылки, имитирующие структуру каталога с модулями puppet чтобы создать подходящие для работы модуля окружение. Файл spec\_helper.rb используется для выполнения puppet в этом окружении.
\item[Rakefile] Rake --- это система автоматизации, похожая на классический make. Этот файл просто описывает действие spec, которое выполняет все тесты в этом модуле. Если вам требуется выполнять какие-то дополнительные задачи до или после работы тестов, то их можно добавить в этот файл.
\end{description}

Наш модуль описывает только один ресурс --- file с определёнными параметрами. Спецификация его работы выглядит так:

\begin{verbatim}
require 'spec_helper'

describe 'motd' do
  it do
    should contain_file('/etc/motd').with({
      'ensure' => 'present',
      'owner'  => 'root',
      'group'  => 'root',
      'mode'   => '0644',
    })
  end

  it do
    should contain_file('/etc/motd').with_content('Hello!')
  end

end
\end{verbatim}

Сначала мы проверяем, что каталог содержит ресурс file с указанными параметрами, затем проверяем, что содержимое этого файла соответствует задаче. Запустим тест и посмотрим на результат.

\begin{verbatim}
> rspec -f doc --color spec/classes/init_spec.rb 

motd
  should contain File[/etc/motd] with ensure => "present", owner => "root",
                                      group => "root" and mode => "0644"
  should contain File[/etc/motd] with content => "Hello!"

Finished in 0.0827 seconds
2 examples, 0 failures
\end{verbatim}

Оба теста завершено успешно.

Тесты также можно запустить используя rake благодаря задаче, которая описана в Rakefile.

\begin{verbatim}
rake spec
\end{verbatim}

Чтобы задать с какими параметрами rspec должен запускаться можно создать файл \textbf{.rspec} в корне этого модуля с таким содержанием:

\begin{verbatim}
--colour
--format documentation
\end{verbatim}

Или можно передать дополнительные параметры, переопределяющие установленные, если нужно один раз выполнить тесты с другими параметрами:

\begin{verbatim}
rake spec SPEC_OPTS="--format html"
\end{verbatim}

Теперь предположим, что при создании этого модуля была допущена ошибка и вместо строки 'Hello!' передаётся строка 'Helo!'. Посмотрим на результат тестов.

\begin{verbatim}
> rake spec
/usr/bin/ruby1.9.1 -S rspec spec/classes/init_spec.rb

motd
  should contain File[/etc/motd] with ensure => "present", owner => "root",
                                      group => "root" and mode => "0644"
  should contain File[/etc/motd] with content => "Hello!" (FAILED - 1)

Failures:

  1) motd 
     Failure/Error: should contain_file('/etc/motd').with_content('Hello!')
       expected that the catalogue would contain File[/etc/motd]
       with content set to `"Hello!"` but it is set to `"Helo!"` in the catalogue
     # ./spec/classes/init_spec.rb:15:in `block (2 levels) in <top (required)>'

Finished in 0.08056 seconds
2 examples, 1 failure

Failed examples:

rspec ./spec/classes/init_spec.rb:14 # motd 
rake aborted!
\end{verbatim}

Один из тестов провален, потому что содержимое файла не соответствует ожидаемому, но остальные параметры файла правильные и первый тест успешен. Таким образом такая мелкая ошибка, на поиск которой зачастую может быть потрачено много часов и нервов, была обнаружена еще до начала настоящего тестирование манифеста.
