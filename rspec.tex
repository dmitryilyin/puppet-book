\chapter{Unit тесты Puppet при помощи Rspec}

\section{Разработка через тестирование}

Rspec --- фреймворк для тестирования поведения классов Ruby. Тестирование поведения это практика программирования, при которой перед написанием самой программы создаётся набор тестов, описывающих её поведение в разных режимах и примерах использования. Эти тесты получаются болше похожими на спецификацию или даже документацию, описывающую работу программы.

После написания тестовс они, естественно, не проходят, потому что программы еще не существует и можно переходить к этапу написания самой программы так, чтобы все тесты слали успешными. Обычно процесс разработки построен итеративно. Сначала реализуются тесты для самых базовых возможностей программы, реализуются эти возможности, потом пишутся тесты для дополнительных более продвинутых возможностей и приступают к их реализации. Таким образом на каждой итерации получается работающий продукт, проходящий все тесты и готовый к использованию, но, возможно, имеющий еще не все возможности, которые планировалось. При каждой последующие итерации всегда выполняются все тесты и происходит обнаружение всех возможных регрессий.

Такой процесс разработки лежит в основе методики Agile и сильно отличается от классической модели, при которой сначала планируются все возможности, разрабатывается архитектура, реализуются все компоненты и потом тестируются, что делает довольно сложным внесение изменений в структуру проекта, и работающая версия получается только на последних этапах работы.

Тестирование поведения --- расширение концепции разработки через тестирование, при котором делается фокус не не структуру программы и соответствие работы каждого её компонента спецификации, а на поведение всей программы целиком и ее реакции на различные ситуации и исходные данные.

\section{Начало работы с Rspec}

Для начала установим Rspec. Это можно сделать при помощи gem или пакетного менеджера вашего дистрибутива.

\begin{verbatim}
gem install rspec
\end{verbatim}


Для примера разработаем простую программу для сложения чисел. Программа должна складывать все переданные числа и выводить результат.

Описание поведения пишется при помощи специального предметно-ориентированного языка, реализованного на Ruby. При описании используется три ключевых слова: Дано, Если, Тогда. Первое слово --- задание условий, при которых производится тестирование, второе --- некоторое событие или входные данные программы, а третье --- результат, который должна вернуть программа при правильной работе.

Сначала опишем группу тестов при помощи слова \textbf{describe}. Весь связанный с ней блок будет входить в эту группу. Группы могут быть вложенными друг в друга и вместо слова \textbf{describe} может быть использованно слово \textbf{context}, чтобы подчеркнуть разделение тестов на относящиеся к разным условиям. Тесты описываются при помощи ключевого слова \textbf{it}. Описание тестов и контекстов обычно пишут простыми словами, которые были бы понятны не только программистам, но и людям не технических специальностей, но знакомыми с бизнес процессами, делая возможным их участие в составлении описаний программ.

\begin{verbatim}
# encoding: UTF-8
require './sum_numbers'

describe "Складыватель Чисел" do
  it "Должен сначала иметь сумму равную нулю" do
    sn = SumNumbers.new
    sn.out.should == 0
  end
  it "Должен иметь указанную начальную сумму, если её указать" do
    sn = SumNumbers.new(3)
    sn.out.should == 3
  end
  it "Должен скадывать каждое переданное число и хранить сумму" do
	sn = SumNumbers.new
    sn.add(2)
    sn.add(3)
    sn.out.should == 5
  end
  it "Должен складывать числа и начальной суммой" do
    sn = SumNumbers.new(2)
    sn.add(2)
    sn.out.should == 4
  end
end
\end{verbatim}

Теперь попробуем запустить эти тесты и убедимся, что они не проходят.

\begin{verbatim}
rspec --color -f doc sum_numbers_spec.rb 

Складыватель Чисел
  Должен сначала иметь сумму равную нулю (FAILED - 1)
  Должен иметь указанную начальную сумму, если её указать (FAILED - 2)
  Должен скадывать каждое переданное число и хранить сумму (FAILED - 3)
  Должен складывать числа и начальной суммой (FAILED - 4)

Failures:

  1) Складыватель Чисел Должен сначала иметь сумму равную нулю
     Failure/Error: sn = SumNumbers.new
     NameError:
       uninitialized constant SumNumbers
     # ./sum_numbers_spec.rb:6:in `block (2 levels) in <top (required)>'

  2) Складыватель Чисел Должен иметь указанную начальную сумму, если её указать
     Failure/Error: sn = SumNumbers.new(3)
     NameError:
       uninitialized constant SumNumbers
     # ./sum_numbers_spec.rb:10:in `block (2 levels) in <top (required)>'

  3) Складыватель Чисел Должен скадывать каждое переданное число и хранить сумму
     Failure/Error: sn = SumNumbers.new
     NameError:
       uninitialized constant SumNumbers
     # ./sum_numbers_spec.rb:14:in `block (2 levels) in <top (required)>'

  4) Складыватель Чисел Должен складывать числа и начальной суммой
     Failure/Error: sn = SumNumbers.new(2)
     NameError:
       uninitialized constant SumNumbers
     # ./sum_numbers_spec.rb:20:in `block (2 levels) in <top (required)>'

Finished in 0.00215 seconds
4 examples, 4 failures

Failed examples:

rspec ./sum_numbers_spec.rb:5
# Складыватель Чисел Должен сначала иметь сумму равную нулю
rspec ./sum_numbers_spec.rb:9
# Складыватель Чисел Должен иметь указанную начальную сумму, если её указать
rspec ./sum_numbers_spec.rb:13
# Складыватель Чисел Должен скадывать каждое переданное число и хранить сумму
rspec ./sum_numbers_spec.rb:19
# Складыватель Чисел Должен складывать числа и начальной суммой
\end{verbatim}

Теперь напишем саму программу, которя складывает числа.

\begin{verbatim}
class SumNumbers
  def initialize(start_sum = 0)
    @sum = start_sum
  end
  def out
    @sum
  end
  def add(add_num = 0)
    @sum += add_num
  end
end
\end{verbatim}

И запустим снова тесты, чтобы её проверить. Убедимся, что все тесты прошли.

\begin{verbatim}
rspec --color -f doc sum_numbers_spec.rb 

Складыватель Чисел
  Должен сначала иметь сумму равную нулю
  Должен иметь указанную начальную сумму, если её указать
  Должен скадывать каждое переданное число и хранить сумму
  Должен складывать числа и начальной суммой

Finished in 0.00174 seconds
4 examples, 0 failures
\end{verbatim}

Предположим, что мы бы допустили ошибку при написании программы и неправильно реализовали установку начального значения суммы.

\begin{verbatim}
class SumNumbers
  def initialize(start_sum = 0)
    @sum = 0
  end
  def out
    @sum
  end
  def add(add_num = 0)
    @sum += add_num
  end
end
\end{verbatim}

Запустим снова наши тесты и посмотрим на результаты.

\begin{verbatim}
rspec --color -f doc sum_numbers_spec.rb 

Складыватель Чисел
  Должен сначала иметь сумму равную нулю
  Должен иметь указанную начальную сумму, если её указать (FAILED - 1)
  Должен скадывать каждое переданное число и хранить сумму
  Должен складывать числа и начальной суммой (FAILED - 2)

Failures:

  1) Складыватель Чисел Должен иметь указанную начальную сумму, если её указать
     Failure/Error: sn.out.should == 3
       expected: 3
            got: 0 (using ==)
     # ./sum_numbers_spec.rb:11:in `block (2 levels) in <top (required)>'

  2) Складыватель Чисел Должен складывать числа и начальной суммой
     Failure/Error: sn.out.should == 4
       expected: 4
            got: 2 (using ==)
     # ./sum_numbers_spec.rb:22:in `block (2 levels) in <top (required)>'

Finished in 0.00168 seconds
4 examples, 2 failures

Failed examples:

rspec ./sum_numbers_spec.rb:9
# Складыватель Чисел Должен иметь указанную начальную сумму, если её указать
rspec ./sum_numbers_spec.rb:19
# Складыватель Чисел Должен складывать числа и начальной суммой
\end{verbatim}

Теперь было провалено только 2 теста. Именно те, которые использовали установку начального значения.

\section{Совпадения и ожидания}

При проверке соответствия результатов выполнения программы правильному значению используются такие понятия как Expectations (Ожидания) и Matchers (Совпадения). Expectations устанавливаются автоматически при установке rspec и добавляют два метода каждому Ruby объекту, которые используются для сравнения его значения с правильным.

\begin{verbatim}
foo.should eq(bar)
foo.should_not eq(bar)
\end{verbatim}

Можно использовать альтернативный синтаксис

\begin{verbatim}
foo.should eq(bar)
foo.should_not eq(bar)  
\end{verbatim}

Аргументом для этих объектов должено быть Совпадение. Эти объекты определяют разные условия, которые могут быть ожидаемы или не ожидаемы при проверке. Среди встроенных совпадений можно найти равенство, неравенство, нахождение в массиве, принадлежность классу и многие другие, список которых и примеры использования можно посмотреть здесь \url{https://github.com/rspec/rspec-expectations}.

\section{Rspec для Puppet}

Тестирование поведения модулей и манифестов puppet не предусматривает их запуск на реальной или виртуальной системе, а описывает, какие ресурсы и с какими параметрами должны быть собраны в каталог, который затем был бы исполнен на управляемой системе. Таким образом проверяется логика, реализуемая модулем и манифестом посредством проверки соответствия результатов их работы эталонным результатам в указанных условиях с определёнными параметрами.

Например можно убедиться, что на Debian системе будет установлен пакет apache и создан файл \texttt{/etc/apache/sites-aviable/mysite.ru.conf} c правильным описанием этого сайта и создана символическая ссылка на этот файл /etc/apache/sites-enabled/mysite.ru.conf. А при запуске на Red Hat системе вместо этого будет установлен пакет httpd и создан файл \texttt{/etc/httpd/conf.d/mysite.ru/conf}.

Такой тест совершенно не гарантирует, что все эти действия действительно будут выполнены правильно, он только проверяет правильность постановки задачи для исполнения на управляемой системе. Процесс может завершиться неудачно по множеству причин, которые не зависят от правильности написания этого манифеста.

Хотя использование методов разработки через тестирование может показаться бесполезным для не очень сложных модулей и манифестов, при увеличении количества управляемых ресурсов, появлении сложных условий и логики, поддержке многих конфигураций и версий операционных систем и, особенно, при дальнейшей доработке существующих манифестов отладка и тестирование станет очень сложным. Поэтому Unit тесты, хотя и не заменяют проверки на тестовых системах, позволят определить большинство логических ошибок, регрессий и несоответствий спецификации еще до начала тестирования. Кроме этого хорошо написаные тесты также выполняют функцию "исполняемой документации", описывая поведение каждого манифеста на языке, который легко понятен не только компьютеру, но и человеку.

Чтобы использовать Rspec для тестирования модулей Puppet нужно сначала установать пакет rspec-puppet, который можно найти здесь \url{http://rspec-puppet.com}. Кроме него потребуется установить сам puppet и дополнительный пакет puppetlabs\_spec\_helper.

\begin{verbatim}
gem install rspec-puppet
gem install puppetlabs_spec_helper
gem install puppet
\end{verbatim}

Теперь создадим простой модуль puppet и тесты для него. Предположим, что у нас есть модуль для управления файлом \textbf{/etc/issue}, состоящий из одного класса.

\begin{verbatim}
class motd {
  file { '/etc/motd' :
    ensure  => present,
    owner   => 'root',
    group   => 'root',
    mode    => '0644',
    content => 'Hello!',
  }
}
\end{verbatim}

Оформим этот класс как модуль, создав соответствующую структуру каталогов и положив этот класс в файл \textbf{init.pp} в каталоге manifests.

\dirtree{%
.1 modules.
.2 motd.
.3 manifests.
.4 init.pp.
.3 tests.
.3 lib.
.3 files.
.3 templates.
}

Теперь воспользуемся скриптом \textbf{rspec-puppet-init} перейдя в каталог этого модуля чтобы создать необходимую для работы Unit тестов структуру файлов и каталогов.

\begin{verbatim}
 + spec/
 + spec/classes/
 + spec/defines/
 + spec/functions/
 + spec/hosts/
 + spec/fixtures/
 + spec/fixtures/manifests/
 + spec/fixtures/modules/
 + spec/fixtures/modules/motd/
 + spec/fixtures/manifests/site.pp
 + spec/fixtures/modules/motd/manifests
 + spec/spec_helper.rb
 + Rakefile
\end{verbatim}

\begin{description}
\item[classes] В этом каталоге нужно размещать файлы, описывающие поведение классов и параметрических классов.
\item[defines] В этом каталоге нужно размещать описание поведения определений.
\item[hosts] Здесь размещаются описания работы модуля на разных системах.
\item[functions] Описания работы специальных функций, если модуль их содержит.
\item[fixtures] Здесь создаются символические ссылки, имитирующие структуру каталога с модулями puppet чтобы создать подходящие для работы модуля окружение. Файл spec\_helper.rb используется для выполнения puppet в этом окружении.
\item[Rakefile] Rake --- это система автоматизации, похожая на классический make или fabric, используемый в python, или ant для Java. Этот файл просто описывает действие spec, которое выполняет все тесты в этом модуле. Если вам требуется выполнять какие-то дополнительные задачи до или после работы тестов, то их можно добавить в этот файл.
\end{description}

Наш модуль описывает только один ресурс --- file с определёнными параметрами. Спецификация его работы выглядит так:

\begin{verbatim}
require 'spec_helper'

describe 'motd' do
  it do
    should contain_file('/etc/motd').with({
      'ensure' => 'present',
      'owner'  => 'root',
      'group'  => 'root',
      'mode'   => '0644',
    })
  end

  it do
    should contain_file('/etc/motd').with_content('Hello!')
  end

end
\end{verbatim}

Сначала мы проверяем, что каталог содержит ресурс file с указанными параметрами, затем проверяем, что содержимое этого файла соответствует задаче. Запустим тест и посмотрим на результат.

\begin{verbatim}
> rspec -f doc --color spec/classes/init_spec.rb 

motd
  should contain File[/etc/motd] with ensure => "present", owner => "root",
                                      group => "root" and mode => "0644"
  should contain File[/etc/motd] with content => "Hello!"

Finished in 0.0827 seconds
2 examples, 0 failures
\end{verbatim}

Оба теста завершено успешно.

Тесты также можно запустить используя rake благодаря задаче, которая описана в Rakefile.

\begin{verbatim}
rake spec
\end{verbatim}

Чтобы задать с какими параметрами rspec должен запускаться можно создать файл \textbf{.rspec} в корне этого модуля с таким содержанием:

\begin{verbatim}
--colour
--format documentation
\end{verbatim}

Или можно передать дополнительные параметры, переопределяющие установленные, если нужно один раз выполнить тесты с другими параметрами:

\begin{verbatim}
rake spec SPEC_OPTS="--format html"
\end{verbatim}

Теперь предположим, что при создании этого модуля была допущена ошибка и вместо строки 'Hello!' передаётся строка 'Helo!'. Посмотрим на результат тестов.

\begin{verbatim}
> rake spec
/usr/bin/ruby1.9.1 -S rspec spec/classes/init_spec.rb

motd
  should contain File[/etc/motd] with ensure => "present", owner => "root",
                                      group => "root" and mode => "0644"
  should contain File[/etc/motd] with content => "Hello!" (FAILED - 1)

Failures:

  1) motd 
     Failure/Error: should contain_file('/etc/motd').with_content('Hello!')
       expected that the catalogue would contain File[/etc/motd]
       with content set to `"Hello!"` but it is set to `"Helo!"` in the catalogue
     # ./spec/classes/init_spec.rb:15:in `block (2 levels) in <top (required)>'

Finished in 0.08056 seconds
2 examples, 1 failure

Failed examples:

rspec ./spec/classes/init_spec.rb:14 # motd 
rake aborted!
\end{verbatim}

Один из тестов провален, потому что содержимое файла не соответствует ожидаемому, но остальные параметры файла правильные и первый тест успешен. Таким образом такая мелкая ошибка, на поиск которой зачастую может быть потрачено много часов и нервов, была обнаружена еще до начала настоящего тестирование манифеста.
